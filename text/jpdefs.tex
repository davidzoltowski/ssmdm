%%%%%%%%%%%%%%%%%%%%%%%%%%%%%%%%%%%%%%%%%%%%%%%%%%%%%%%%%%%%%%%%%%%%%%%%%%%
%%%
%%% Helpful definitions for TeX
%%%  Eero Simoncelli, 6/90.
%%%
%%%%%%%%%%%%%%%%%%%%%%%%%%%%%%%%%%%%%%%%%%%%%%%%%%%%%%%%%%%%%%%%%%%%%%%%%%%

\long\def\comment#1{}
\long\def\nocomment#1{#1}   %to easily re-insert commented text
\long\def\mcomment#1{
	\marginpar[\singlesp\footnotesize #1 \fbox{\rule{1ex}{0ex}\rule{0ex}{1ex}}]
		  {\singlesp\footnotesize \fbox{\rule{1ex}{0ex}\rule{0ex}{1ex}} #1}}

%%% Set the spacing between lines: 1.0 means single-space.
%%% Must call a textsize command to get the new baselinestretch to
%%% work.  Thus this function resets the text size to \normalsize.
\def\setspacing#1{\renewcommand\baselinestretch{#1}\large\normalsize}

%%%  Use before 1st paragraph of a section to force an indentation.
%%% Correct solution is to redefine the sectioning commands in the
%%% appropriate style files (eg, art11.sty).
\newcommand{\yesindent}{\hspace*{\parindent}}    %%{\hskip \parindent}

\newcommand\doublesp{\renewcommand\baselinestretch{1.8}\large\normalsize}
\newcommand\singlesp{\renewcommand\baselinestretch{1.1}\large\normalsize}
\newcommand\halfsp{\renewcommand\baselinestretch{1.5}\large\normalsize}

\newenvironment{ct}
	{\begin{center}\begin{tabular}}
	{\end{tabular}\end{center}}

%%% Non-numbered subsections, with table of contents entries
\def\mychapter#1{\def\baselinestretch{1.0}\chapter{#1}\vskip 0.25in}
\def\mychap#1{\addcontentsline{toc}{chapter}{#1}
		\def\baselinestretch{1.0}\chapter*{#1}\vskip 0.25in}
\def\mysec#1{\addcontentsline{toc}{section}{$\circ$\hskip 0.4em #1}
		\section*{#1}}
\def\mysubsec#1{\addcontentsline{toc}{subsection}{$\circ$\hskip 0.4em #1} 
		\subsection*{#1}}
\def\mysubsubsec#1{\addcontentsline{toc}{subsubsection}{$\circ$\hskip 0.4em #1}
		\subsubsection*{#1}}

%%% A macro for figures to be MANUALLY pasted.  The arguments are the figure
%%% height, the figure label, and the figure caption.  The macro makes
%%% space for the figure, prints the label in center of the space (as
%%% an identifier), and prints the caption.
\long\def\myfig#1#2#3{
	\vbox to #1{\vfil \center{#2} \vfil} %Print label in box
	\caption{#3}
	\label{#2}}

%%% Macro for vertically aligning a subfigure and its label.
%%% Example usage: \sublabel{(a)}{\psfig{ ... }}
\def\sublabel#1#2{ \begin{tabular}{c} #2 \\  #1 \end{tabular}}

% %%% New version of \caption puts things in smaller type, single-spaced 
% %%% and indents them to set them off more from the text.
% \makeatletter
% \long\def\@makecaption#1#2{
% 	\vskip 0.8ex
% 	\setbox\@tempboxa\hbox{\small {\bf #1:} #2}
%         \parindent 1.5em  %% How can we use the global value of this???
% 	\dimen0=\hsize
% 	\advance\dimen0 by -3em
% 	\ifdim \wd\@tempboxa >\dimen0
% 		\hbox to \hsize{
% 			\parindent 0em
% 			\hfil 
% 			\parbox{\dimen0}{\def\baselinestretch{0.96}\small
% 				{\bf #1.} #2
% 				%%\unhbox\@tempboxa
% 				} 
% 			\hfil}
% 	\else \hbox to \hsize{\hfil \box\@tempboxa \hfil}
% 	\fi
% 	}
% \makeatother

%%%% original version of caption
%\long\def\@makecaption#1#2{
%   \vskip 10pt 
%   \setbox\@tempboxa\hbox{#1: #2}  
%   \ifdim \wd\@tempboxa >\hsize   % IF longer than one line:
%       \unhbox\@tempboxa\par      %   THEN set as ordinary paragraph.
%     \else                        %   ELSE  center.
%       \hbox to\hsize{\hfil\box\@tempboxa\hfil}  
%   \fi}

%%% Footnotes without numbers, modified from latex.tex definition of
%%% footnotetext.  NOTE: Can use \thanks if doing this on the title
%%% page. 
\makeatletter
\long\def\barenote#1{
    \insert\footins{\footnotesize
    \interlinepenalty\interfootnotelinepenalty 
    \splittopskip\footnotesep
    \splitmaxdepth \dp\strutbox \floatingpenalty \@MM
    \hsize\columnwidth \@parboxrestore
    {\rule{\z@}{\footnotesep}\ignorespaces
	      % indent
      #1\strut}}}
\makeatother

\def\bulletnote#1{\barenote{$\bullet$ #1}}
        
%%% Math definitions:
%\newtheorem{prop}{Proposition}
%\newenvironment{proof}{{\sl Proof: }}{\rule{0.07in}{0.13in}}

%%%p.100 Latex for \fbox with \rule
%%%Hollow: {\fbox{\rule{0.07in}{0in}\rule{0in}{0.14in}}}

%%% Inserting text into multi-line formulas.  See p. 193, Tex manual.
%%% Can also use \rm \noindent instead of \hbox, allowing longer lines.
%%\def\mtext#1{\noalign{\hbox{#1} \smallskip \medskip}}
\long\def\mtext#1{\noalign{\rm \noindent{#1}}}

%%% Insert space after or before or in an equation
%% \vbox to \medskipamount{\vfil}
%\def\eqsmallskip{\hbox{\smallskip }\nonumber\\}
%\def\eqmedskip{\hbox{\medskip} \nonumber \\}  
%\def\eqbigskip{\hbox{\bigskip} \nonumber \\} 
\def\eqsmallskip{\rule[\baselineskip]{0pt}{\smallskipamount}}
\def\eqmedskip{\rule[\baselineskip]{0pt}{\medskipamount}}
\def\eqbigskip{\rule[\baselineskip]{0pt}{\bigskipamount}}

%%% Alternative: use a strut (see p. 100 of LaTeX manual)

%%% Display fraction in textstyle (ie, with slash rather than
%%% horizontal bar)
\def\sfrac#1#2{{\textstyle \frac{#1}{#2}}}   

%% Format in tabbing (paragraph) mode, producing a box that is shrunk
%% to fit the paragraph.  First arg is margin, second is paragraph.
%% NOTE: minipage of tabbing fits paragraph, see LaTeX, p. 100.
\makeatletter
\long\def\shrinkbox{\@ifnextchar[{\@mgnshrinkbox}{\@shrinkbox}}

\def\@shrinkbox#1{
	\begin{minipage}{\columnwidth}\begin{tabbing}
	#1
	\end{tabbing}\end{minipage}}

\def\@mgnshrinkbox[#1]#2{
   \setbox\@tempboxa\hbox{\unskip\begin{minipage}{\columnwidth}\begin{tabbing}
		#2
		\end{tabbing}\end{minipage}\unskip}
   \dimen0=\wd\@tempboxa
   \advance\dimen0 by #1
   %%\advance\dimen0 by #1
   \dimen1=\ht\@tempboxa
   \advance\dimen1 by #1
   \advance\dimen1 by #1
   \makebox[\dimen0]{\vbox to \dimen1{\vfil \box\@tempboxa \vfil}}}
\makeatother

%  \F for the Fourieroperator
\newcommand{\F}{\mbox{$\cal F$}}

%  \R and \C   The set of Real and Complex numbers
%  may be used in equation or text mode
%  LaTeX has \Re but no \Co; define both in the traditional form (non script)
%\newcommand{\R}{\mbox{\rule[0ex]{.2ex}{1.56ex} \hspace{-1.6ex} \bf R}}
\newcommand{\R}{\mbox{I \hspace{-2ex} {\bf R}}} 
\newcommand{\C}{\mbox{   % mbox removes italics in equation mode
                \rule[.125ex]{0ex}{1.1ex}  % take out second line for now
                \hspace{-1.2ex}
                \rule[.0625ex]{.2ex}{1.4ex}
                \hspace{-2.4ex}
                \bf C
                }}
\newcommand{\Z}{\mbox{Z\hspace{-2ex}{\bf Z}}}

\def\etal{{\sl et~al.}}

%%new caption macro
%\makeatletter
%\long\def\@makecaption#1#2{
%        \vskip 10pt
%        \setbox\@tempboxa\hbox{\small {\bf #1:} #2}
%        \parindent .75em  %% How can we use the global value of this???
%        \dimen0=\hsize
%        \advance\dimen0 by -2\parindent
%        \ifdim \wd\@tempboxa >\dimen0
%                \hbox to \hsize{
%                        \hfil
%                        \parbox{\dimen0}{\def\baselinestretch{1.05}\small
%                                {\bf #1:} #2
%                                %%\unhbox\@tempboxa
%                                }
%                        \hfil}
%        \else \hbox to \hsize{\hfil \box\@tempboxa \hfil}
%        \fi}
%\makeatother

% ============================================================
% More macros (jpdefs.tex)

\newcommand{\asection}[1]{
        \addtocounter{section}{1}
        \setcounter{subsection}{0}
        \section*{Appendix \Alph{section}: #1}
        \addcontentsline{toc}{section}{Appendix \Alph{section}: #1}}

\newcommand{\asubsection}[1]{
        \addtocounter{subsection}{1}
        \subsection*{\Alph{section}.\arabic{subsection} #1}
\addcontentsline{toc}{subsection}{\Alph{section}.\arabic{subsection} #1}}


\newtheorem{theorem}{Theorem}
\newtheorem{lemma}[theorem]{Lemma}

% ---------- Symbols for ``Reals'', ``Complex Numbers'', etc. ----
\newcommand{\Reals}{{\sf R\hspace*{-0.9ex}\rule{0.15ex}%
{1.5ex}\hspace*{0.9ex}}}
\newcommand{\N}{{\sf N\hspace*{-1.0ex}\rule{0.15ex}%
{1.3ex}\hspace*{1.0ex}}}
\newcommand{\Q}{{\sf Q\hspace*{-1.1ex}\rule{0.15ex}%
{1.5ex}\hspace*{1.1ex}}}
\newcommand{\Complexs}{{\sf C\hspace*{-0.9ex}\rule{0.15ex}%
{1.3ex}\hspace*{0.9ex}}}


% --- Standard Math definitiones -----
\newcommand{\RR}{I\!\!R} %real numbers 
\newcommand{\Nat}{I\!\!N} %natural numbers 
\newcommand{\CC}{I\!\!\!\!C} %complex numbers

% ------- Others -------------------------------
\newcommand{\deriv}[2]{\frac{d #1}{d #2}}
\newcommand{\E}{\mathbb{E}}
\newcommand{\D}[2]{\frac{\partial #1}{\partial #2}}
\newcommand{\DD}[2]{\frac{\partial ^2 #1}{\partial #2 ^2}}
\newcommand{\Di}[2]{\frac{\partial ^i #1}{\partial #2 ^i}}
\newcommand{\evalat}[1]{\left.#1\right|}
\newcommand{\parderiv}[2]{\frac{\partial #1}{\partial{#2}}}
\newcommand{\parderivtwo}[3]{\frac{\partial^2 #1}{\partial{#2}\partial{#3}}}
\newcommand{\parDeriv}[3]{\frac{\partial^{#3} #1}{\partial{#2}^{#3}}}
\newcommand{\parwrt}[1]{\frac{\partial}{\partial{#1}}}
\newcommand{\parpowrt}[2]{\frac{\partial^{#1}}{\partial {#2}^{#1}}}
\newcommand{\partwowrt}[2]{\frac{\partial^{2}}{\partial {#2} \partial{#1}}}
\newcommand{\inv}{^{-1}}
\newcommand{\trp}{{^\top}} % transpose
\newcommand{\onehlf}{\frac{1}{2}}
\newcommand{\tonehlf}{\tfrac{1}{2}}
\renewcommand{\eqref}[1]{eq.~\ref{eq:#1}}
\newcommand{\Nrm}{\mathcal{N}}
\newcommand{\Tr}{\mathrm{Tr}}
\renewcommand{\inv}{^{-1}}
\newcommand{\diag}{\mathrm{diag}}
%\newcommand{\evalat}[1]{\left.#1\right|}

\DeclareMathOperator*{\argmin}{arg\,min}
\DeclareMathOperator*{\argmax}{arg\,max}

\renewcommand{\eqref}[1]{eq.~\ref{eq:#1}}
\newcommand{\figef}[1]{Fig.~\ref{fig:#1}}
